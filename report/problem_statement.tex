\section{Problem Statement}

In Kubernetes deployments, there can be enormous variables which need to be set before container deployment. This task is carried out by dev-ops colleagues. Manual tweaking and adjustment of variables are required to get a container up and running. 
This can lead to errors, inconsistencies, and downtime, particularly in dynamic environments where configuration changes are frequent. The problem statement revolves around the need for a robust and automated solution to intelligently inject environment variables into Kubernetes clusters and automatically trigger pod restarts upon detecting changes or updates to the configuration.

\bigskip

\textbf{Key challenges include:}

\begin{enumerate}
    \item \textbf{Manual Configuration Management:} Manually updating environment variables in infrastructure is time-consuming, error-prone, and can result in configuration drifts and inconsistencies across pods.
    
    \item \textbf{Dynamic Environment:} In dynamic Kubernetes environments where pods scale up, scale down, or migrate across nodes, ensuring that all pods have access to the latest environment variables becomes challenging.
    
    \item \textbf{Application Reliability:} Changes to environment variables, such as API keys or database credentials, may require pod restarts to take effect, impacting application availability and reliability if not handled efficiently.
    
    \item \textbf{Scalability:} As Kubernetes deployments scale to accommodate varying workloads, the process of injecting and updating environment variables must scale accordingly to meet the demands of dynamic environments.
\end{enumerate}

Addressing these challenges requires the development of an automated solution that integrates with Kubernetes deployments, intelligently manages environment variables, and ensures the reliability, security, and scalability of applications running in Kubernetes clusters.
