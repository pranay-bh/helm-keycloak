\section{Background and Motivation}

\subsection{HashiCorp Vault}

\textbf{Introduction:}
HashiCorp Vault is a popular open-source tool for managing secrets and sensitive data in modern cloud-native environments. It provides a centralized platform for securely storing, accessing, and managing secrets such as API keys, passwords, certificates, and encryption keys. Vault offers a robust set of features and capabilities designed to address the challenges of secrets management in dynamic and distributed systems. \cite{vault}

\textbf{Use and Benefits:}
The primary use of HashiCorp Vault is to securely manage secrets and sensitive data in cloud-native applications and infrastructure. By centralizing the storage of secrets, Vault helps organizations improve security, compliance, and operational efficiency.

\textbf{Thanks to Built-in APIs} which allow developers and operators to interact with Vault programmatically. These APIs provide access to Vault's functionality, including the ability to retrieve, create, update, and delete secrets.  \cite{vault-api}

\textbf{Key/Value Secrets Engine API} is the API allows users to store and retrieve secrets as key/value pairs within Vault's hierarchical data store. This also provides metadata, which defines what is the current of secret of defiled in each Key/Value secret engine. \cite{vault-kv}

Additionally, Vault provide a User interface, which developers can use to add, modify, delete and update the secrets which are inside kubernetes container. \cite{vault-ui}

In summary, HashiCorp Vault is a powerful tool for managing variables in cloud-native environments. Its rich set of features and flexible APIs makes it an essential component for our architecture.

\subsection{Liveness Probes}

\textbf{Functionality of Liveness Probe}

\textbf{What is Liveness Probe:}
In Kubernetes, a liveness probe is a mechanism used to determine the health of a containerized application running in a pod. It periodically checks the application's state to ensure that it is running as expected. The liveness probe performs this check by sending requests to a specified endpoint within the container and evaluating the response. If the application responds successfully, indicating that it is healthy, the liveness probe considers the container to be in a good state. However, if the application fails to respond or returns an error, indicating that it is unhealthy, the liveness probe takes action, such as restarting the container. \cite{kubernetes-probes} \cite{zahorscak}

\textbf{Benefits:}
The use of liveness probes offers several benefits for managing containerized applications in Kubernetes deployments:

\begin{itemize}
    \item \textbf{Improved Reliability:} By continuously monitoring the health of containerized applications, liveness probes help ensure that unhealthy containers are detected and replaced promptly, minimizing downtime and enhancing application reliability. \cite{odabasi}
    
    \item \textbf{Automatic Recovery:} Liveness probes facilitate automatic recovery from application failures by restarting unhealthy containers, thereby maintaining the desired state of the application and preserving service availability. \cite{kubernetes-probes}
    
    \item \textbf{Scalability:} Liveness probes enable Kubernetes clusters to dynamically scale resources based on workload demands, as unhealthy containers are replaced with healthy ones, allowing the cluster to adapt to changing conditions efficiently. \cite{zahorscak}
    
    \item \textbf{Simplified Operations:} By automating the detection and recovery of unhealthy containers, liveness probes reduce the need for manual intervention and streamline operations, making it easier to manage large-scale Kubernetes deployments. \cite{zahorscak}
\end{itemize}

\textbf{Parameters:}
Liveness probes in Kubernetes are configured using several parameters that define how the probe operates and when it should take action:

\begin{itemize}
    \item \textbf{Initial Delay:} This parameter specifies the amount of time to wait after the container starts before performing the first liveness probe. It allows the application to initialize before being checked for health. \cite{kubernetes-probes}
    
    \item \textbf{Period:} The period parameter defines the frequency at which the liveness probe should be executed, indicating how often the application's health should be assessed.\cite{kubernetes-probes}
    
    \item \textbf{Timeout:} This parameter sets the maximum amount of time the probe should wait for a response from the application. If the application fails to respond within this timeframe, the probe considers it unhealthy.\cite{kubernetes-probes}
    
    \item \textbf{Failure Threshold:} Conversely, the failure threshold parameter defines the number of consecutive failed probe results that indicate the application is unhealthy and should be restarted.\cite{kubernetes-probes}
\end{itemize}

\textbf{Leveraging Liveness Probe for Conditional Restart of Cluster:}
By leveraging the functionality of liveness probes, Kubernetes deployments can implement conditional restarts of clusters based on the health status of containerized applications. We will use liveness probe to verify if the secret version of a key/vault secret engine matches to current deployment version. If both version matched, we will classify pod as healthy. In case of mismatch pod should restart itself automatically and fetch latest version of secrets from the vault. 
